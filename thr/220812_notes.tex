\documentclass[a4paper,10pt]{article}

\usepackage[
	left=1cm,right=1cm,top=1cm,bottom=2cm,
    papersize={250mm,350mm}
]{geometry}


\usepackage[english]{babel}
\usepackage[utf8]{inputenc}

\usepackage{amsmath}
\usepackage{amssymb}
\usepackage{physics}
\usepackage{dsfont}

\usepackage[squaren]{SIunits}
\usepackage{graphicx}
\usepackage{xcolor}
\usepackage{verbatim}

\newcommand{\mi}{\mathrm{i}}
\newcommand{\suml}{\sum\limits}
\newcommand{\prodl}{\prod\limits}
\newcommand{\intl}{\int\limits}

\newcommand{\gammar}{\gamma_{\text{r}}}
\newcommand{\gammai}{\gamma_{\text{i}}}

\begin{document}

\section{1-Dimensional Examples}

\subsection{Transcritical Bifurcation Model}

\begin{align}
	a
	&\in
	\mathbb{R}\setminus\Bqty{0}
\end{align}

\begin{align}
	\dot{x}
	&=
	ax
	-
	x^2
\end{align}

\begin{align}
	x(t)
	&=
	\frac{
		a	
	}
	{
		1
		-
		e^{-ac_1-at}
	}
\end{align}

\begin{align}
	x(0)
	&=
	x_0
	=
	\frac{
		a	
	}
	{
		1
		-
		e^{-ac_1}
	}
\end{align}

$x_0\neq 0$, $x_0\neq a$ 

\begin{align}
	c_1
	&=
	a^{-1}
	\ln\pqty{\frac{x_0}{x_0-a}}
\end{align}

\begin{align}
	x(t)
	&=
	\frac{
		a	
	}
	{
		1
		-
		\frac{x_0-a}{x_0}		
		e^{-at}
	}
\end{align}

\begin{align}
	\dot{x}
	&=
	f(x)
\end{align}

\begin{align}
	0	
	&=
	f(\bar{x})
	=
	a\bar{x}
	-	
	\bar{x}^2
\end{align}

\begin{align}
	\bar{x}_0
	&=
	0
	,\quad
	\bar{x}_a
	=
	a
\end{align}

\begin{align}
	\mathsf{J}_{f(x)}
	&=
	a-2\bar{x}
\end{align}

\begin{align}
	\varsigma
	&=
	a-2\bar{x}
\end{align}

\begin{align}
	\varsigma_0
	&=
	a
\\
	\varsigma_a
	&=
	-a
\end{align}

\subsubsection{Pyragas Control}

\begin{align}
	\dot{x}
	&=
	a
	x
	-
	x^2
	+
	K
	(x-x_\tau)	
\end{align}

\begin{align}
	\dot{x}
	&=
	f(x,x_\tau)
\end{align}

\begin{align}
	0
	&=
	f(\bar{x},\bar{x}_\tau)
	=
	a\bar{x}
	-
	\bar{x}^2
	+
	K
	(\bar{x}-\bar{x}_\tau)	
\end{align}

\begin{align}
	\delta
	\dot{x}
	&=
	\pqty{
		a
		-
		2\bar{x}
		+
		K
	}
	\delta x
	-
	K
	\delta x_\tau
\end{align}

\subsubsection{Wiener Process}

\begin{align}
	\expval{\xi(t)}
	&=
	0
\\
	\expval{\xi(t)\xi(t^\prime)}
	&=
	\delta(t-t^\prime)
\end{align}

\begin{align}
	\dot{X}
	&=
	aX
	-
	X^2
	+
	D_0
	\xi(t)
\end{align}

\begin{align}
	\sigma_0
	&=
	\frac{
		D_0^2
	}{2}
\end{align}

\begin{align}
	\dot{X}
	&=
	f(X)
	+
	D_0
	\xi(t)
\end{align}

\begin{align}
	X
	&\in
	\Omega
	\subseteq
	\mathbb{R}
\end{align}

\begin{align}
	\partial_t
	\varrho(x,t)
	&=
	-
	\partial_x
	f(x)
	\varrho(x,t)
	+
	\sigma_0
	\partial_x^2
	\varrho(x,t)	
\end{align}

\begin{align}
	\partial_t
	\varrho(x,t)
	&=
	(2x-a)
	\varrho(x,t)
	+
	(x^2-ax)
	\partial_x
	\varrho(x,t)
	+
	\sigma_0
	\partial_x^2
	\varrho(x,t)	
\end{align}

\subsubsection{Wiener Process and Pyragas Control}

\begin{align}
	\dot{X}
	&=
	aX
	-
	X^2
	+
	K
	(X-X_\tau)	
	+
	D_0
	\xi(t)
\end{align}

\begin{align}
	\dot{X}
	&=
	f(X,X_\tau)
	+
	D_0
	\xi(t)
\end{align}


\begin{align}
	\partial_t
	\varrho_1(x,t)
	&=
	-
	\partial_x
	\intl_\Omega\dd{x_\tau}
	f(x,x_\tau)
	\varrho_2(x,x_\tau,t,t-\tau)
	+
	\sigma_0
	\partial_x^2
	\varrho_1(x,t)
\\
	\partial_t
	\varrho_2(x,x_\tau,t,t-\tau)
	&=
	-
	\partial_x
	\pqty{
		f(x,x_\tau)
		\varrho_2(x,x_\tau,t,t-\tau)
	}
	+
	\sigma_0
	\partial_x^2
	\varrho_2(x,x_\tau,t,t-\tau)
\\
	&+
	\partial_{x_\tau}
	\varrho_2(x,x_\tau,t,t-\tau)
	\intl_{\Omega}\dd{\dot{x}_\tau}
	\dot{x}_\tau
	\varrho_1(\dot{x}_\tau,t-\tau)
\end{align}

\begin{align}
	\partial_t
	\varrho_1(x,t)
	&=
	(2x-(a+K))
	\varrho_1(x,t)
	+
	(x^2-(a+K)x)
	\partial_x
	\varrho_1(x,t)
\\
	&+
	K	
	\partial_x
	\intl_\Omega\dd{x_\tau}
	x_\tau
	\varrho_2(x,x_\tau,t,t-\tau)
	+
	\sigma_0
	\partial_x^2
	\varrho_1(x,t)
\\
	\partial_t
	\varrho_2(x,x_\tau,t,t-\tau)
	&=
	\pqty{
		(2x-(a+K)x)
		\varrho_2(x,x_\tau,t,t-\tau)
	}
	-
	\pqty{
		(ax-x^2+K(x-x_\tau))
	}
	\partial_x
	\varrho_2(x,x_\tau,t,t-\tau)
\\
	&+
	\sigma_0
	\partial_x^2
	\varrho_2(x,x_\tau,t,t-\tau)
	+
	\partial_{x_\tau}
	\varrho_2(x,x_\tau,t,t-\tau)
	\intl_{\Omega}\dd{\dot{x}_\tau}
	\dot{x}_\tau
	\varrho_1(\dot{x}_\tau,t-\tau)
\end{align}

\newpage

\section{(2,1)-Dimensional Examples}

\subsection{Van der Pol Oscillator}

\begin{align}
	\dot{x}
	&=
	y
\\
	\dot{y}
	&=
	\pqty{\varepsilon-x^2}
	y
	-
	\omega_0^2
	x
\end{align}

\begin{align}
	\dot{x}
	&=
	y
\\
	\dot{y}
	&=
	\pqty{\varepsilon-x^2}
	y
	-
	\omega_0^2
	x	
	+
	K
	(x-x_\tau)
\end{align}

\begin{align}
	\dot{x}
	&=
	y
\\
	\dot{y}
	&=
	\pqty{\varepsilon-x^2}
	y
	-
	\omega_0^2
	x	
	+
	K
	(x-x_\tau)
	+
	D_0
	\xi(t)
\end{align}

\section{2-Dimensional Examples}

\subsection{Stuart–Landau Equation}

\begin{align*}
	\lambda,\alpha,K,D
	&\in\mathbb{R},
	\quad
	\gamma\in\mathbb{C}
\\
	\lambda
	&>0,
	\quad
	\gammar
	>0
\end{align*}

\begin{align}
	\dd{z}
	&=
	\pqty{
		\lambda
		+
		\mi\alpha
		-
		\gamma\abs{z}^2
	}z
	\dd{t}
	+
	K\pqty{z-z_{\tau}}
	\dd{t}
	+
	D
	\dd{W}
\end{align}

\begin{align}
	\dd{z}
	&=
	\pqty{
		\lambda
		+
		\mi\alpha
		-
		\gamma\abs{z}^2
	}z
	\dd{t}	
\end{align}

\begin{align}
	z
	&=
	r e^{\mi\varphi}
	,\quad
	\dd{z}
	=
	e^{\mi\varphi}
	\dd{r}
	+
	\mi
	r
	e^{\mi\varphi}
	\dd{\varphi}	
\end{align}

\begin{align}
	e^{\mi\varphi}
	\dd{r}
	+
	\mi
	r
	e^{\mi\varphi}
	\dd{\varphi}
	&=
	\pqty{\lambda+\mi\alpha-\gamma r^2}
	r e^{\mi\varphi}
	\dd{t}
\\
	\dd{r}
	+
	\mi
	r
	\dd{\varphi}
	&=
	\pqty{
		\lambda r
		+
		\mi\alpha r
		-
		\gammar r^3
		-
		\mi\gammai r^3
	}
	\dd{t}
\end{align}

\begin{align}
	\dd{r}
	&=
	\pqty{\lambda r-\gammar r^3}
	\dd{t}
\\
	\dd{\varphi}
	&=
	\pqty{
		\alpha
		-
		\gammai
		r^2
	}	
	\dd{t}
\end{align}

\begin{align}
	\eta
	&=
	r^{-2}
\\
	\dv{\eta}{r}
	&=
	-2r^{-3}
\end{align}

\begin{align}
	\dd{\eta}
	&=
	\dv{\eta}{r}
	\dd{r}
	=
	-2r^{-3}
	\pqty{\lambda r-\gammar r^3}
	\dd{t}
	=
	-2
	\pqty{\lambda\eta-\gammar }
	\dd{t}
\end{align}

\begin{align}
	\dot{\eta}
	&=
	-2
	\lambda\eta
	+
	2
	\gammar
\end{align}

\begin{align}
	\eta(t)
	&=
	c_1
	e^{-2\lambda t}
	+
	\frac{\gammar}{\lambda}
\end{align}

\begin{align}
	\eta_0
	&=
	\eta(0)
	=
	c_1
	+
	\frac{\gammar}{\lambda}
\\
	c_1
	&=
	\eta_0
	-
	\frac{\gammar}{\lambda}		
\end{align}

\begin{align}
	\eta(t)
	&=
	\pqty{
		\eta_0
		-
		\frac{\gammar}{\lambda}	
	}
	e^{-2\lambda t}
	+
	\frac{\gammar}{\lambda}
\\
	r(t)
	&=
	\frac{1}{
		\sqrt{
			\pqty{
				r_0^{-2}
				-
				\frac{\gammar}{\lambda}	
			}
			e^{-2\lambda t}
			+
			\frac{\gammar}{\lambda}
		}	
	}
\end{align}

\begin{align}
	\dot{\varphi}
	&=
	\alpha
	-
	\frac{\gammai}
	{
		\pqty{
			r_0^{-2}
			-
			\frac{\gammar}{\lambda}	
		}
		e^{-2\lambda t}
		+
		\frac{\gammar}{\lambda}	
	}
\end{align}

\begin{align}
	\varphi(t)
	&=
	\varphi(0)
	+
	\intl_{0}^t
	\dd{t^\prime}
	\pqty{
		\alpha
		-
		\frac{\gammai}
		{
			\pqty{
				r_0^{-2}
				-
				\frac{\gammar}{\lambda}	
			}
			e^{-2\lambda t^\prime}
			+
			\frac{\gammar}{\lambda}	
		}
	}
\\
	&=
	\varphi(0)
	+
	\alpha
	t
	-
	\gammai
	\intl_{0}^t
	\dd{t^\prime}
	\frac{1}
	{
		\pqty{
			r_0^{-2}
			-
			\frac{\gammar}{\lambda}	
		}
		e^{-2\lambda t^\prime}
		+
		\frac{\gammar}{\lambda}	
	}
\\
	&=
	\varphi(0)
	+
	\alpha
	t
	-
	\gammai
	\lambda
	r_0^2
	\intl_{0}^t
	\dd{t^\prime}
	\frac{1}
	{
		\pqty{
			\lambda
			-
			\gammar r_0^2	
		}
		e^{-2\lambda t^\prime}
		+
		\gammar r_0^2	
	}
\\
	&=
	\varphi(0)
	+
	\alpha
	t
	-
	\gammai
	\lambda
	r_0^2
	\intl_{0}^t
	\dd{t^\prime}
	\frac{e^{2\lambda t^\prime}}
	{
		\lambda
		-
		\gammar r_0^2
		+
		\gammar r_0^2
		e^{2\lambda t^\prime}
	}
\end{align}

\begin{align}
	u(t^\prime)
	&=
	\lambda
	-
	\gammar r_0^2
	+
	\gammar r_0^2
	e^{2\lambda t^\prime}
\\
	\dv{u}{t^\prime}
	&=
	2\gammar\lambda r_0^2
	e^{2\lambda t^\prime}
\end{align}

\begin{align}
	\varphi(t)
	&=
	\varphi(0)
	+
	\alpha
	t
	-
	\frac{\gammai}{2\gammar}
	\intl_{u(0)}^{u(t)}
	\dd{u}
	\frac{1}{u}
\\
	&=
	\varphi(0)
	+
	\alpha
	t
	-
	\frac{\gammai}{2\gammar}
	\pqty{
		\ln\abs{u(t)}
		-
		\ln\abs{u(0)}
	}
\\
	&=
	\varphi(0)
	+
	\alpha
	t
	-
	\frac{\gammai}{2\gammar}
	\pqty{
		\ln\abs{
			\lambda
			-
			\gammar r_0^2
			+
			\gammar r_0^2
			e^{2\lambda t}		
		}
		-
		\ln\abs{
			\lambda
			-
			\gammar r_0^2
		}
	}
\\
	&=
	\varphi(0)
	+
	\alpha
	t
	-
	\frac{\gammai}{2\gammar}
	\ln\abs{
		1
		+
		\frac{
			\gammar r_0^2
			e^{2\lambda t}
		}
		{
			\lambda
			-
			\gammar r_0^2
		}	
	}
\end{align}


\begin{align}
	\dot{r}
	&=
	\lambda r
	-
	\gammar
	r^3
\\
	\dot{\varphi}
	&=
	\alpha
	-
	\gammai
	r^2
\end{align}

\begin{align}
	\dot{\vb{r}}
	&=
	\vb{f}
	(r,\varphi)
\end{align}

\begin{align}
	0
	&=
	\vb{f}
	(\bar{r},\bar{\varphi})
	=
	\pmqty{
		\lambda \bar{r}
		-
		\gammar
		\bar{r}^3
	\\
		\alpha
		-
		\gammai
		\bar{r}^2
	}
\end{align}

\begin{align}
	\mathsf{J}_{\vb{f}(r,\varphi)}
	&=
	\pmqty{
		\lambda-3\gammar r^2
		&
		0
		\\
		-2\gammai r
		&
		0
	}
\end{align}

\begin{align}
	\varsigma_1
	&=
	\lambda-3\gammar r^2
\\
	\varsigma_2
	&=
	0
\end{align}

\begin{align}
	\dot{x}
	+
	\mi\dot{y}
	&=
	\pqty{
		\lambda
		+
		\mi\alpha
		-
		(
			\gammar
			+
			\mi\gammai		
		)
		(x^2+y^2)
	}
	\pqty{
		x
		+
		\mi y
	}
\\
	&=
	\lambda
	x
	+
	\mi\alpha
	x		
	-
	\gammar
	(x^2+y^2)		
	x		
	-
	\mi\gammai		
	(x^2+y^2)
	x
\\
	&+
	\mi
	\lambda
	y
	-
	\alpha
	y
	-
	\mi
	\gammar
	(x^2+y^2)
	y
	+
	\gammai
	(x^2+y^2)
	y
\end{align}

\begin{align}
	\dot{x}
	&=
	\lambda
	x
	-
	\alpha
	y
	-
	(
		\gammar
		x
		-
		\gammai
		y	
	)
	(x^2+y^2)
\\
	\dot{y}
	&=
	\alpha
	x
	+	
	\lambda
	y		
	-
	(
		\gammai
		x
		+
		\gammar
		y
	)
	(x^2+y^2)
\end{align}

\begin{align}
	\pmqty{
		x\\y	
	}
	\otimes
	\pmqty{
		\gammar\\\gammai	
	}
	&=
	\pmqty{
		\gammar
		x
		-
		\gammai
		y
		\\
		\gammai
		x
		+
		\gammar
		y
	}
\end{align}

\begin{align}
	\dot{\vb{r}}
	&=
	\vb{f}
	(x,y)
\end{align}

\begin{align}
	0
	&=
	\vb{f}
	(\bar{x},\bar{y})
	=
	\pmqty{
		\lambda
		\bar{x}
		-
		\alpha
		\bar{y}
		-
		(
			\gammar
			\bar{x}
			-
			\gammai
			\bar{y}
		)
		(\bar{x}^2+\bar{y}^2)
	\\
		\alpha
		\bar{x}
		+	
		\lambda
		\bar{y}
		-
		(
			\gammai
			\bar{x}
			+
			\gammar
			\bar{y}
		)
		(\bar{x}^2+\bar{y}^2)
	}
\end{align}

\begin{align}
	(\bar{x}_1,\bar{y}_1)
	&=
	(0,0)
\end{align}

\begin{align}
	(\bar{x}_2,\bar{y}_2)
	&=
	\sqrt{\frac{\lambda}{\gammar}}
	(\cos(\bar{\varphi}),\sin(\bar{\varphi})),
	\quad
	\bar{\varphi}\in\left[0,2\pi\right[
\end{align}

\begin{align}
	\mathsf{J}_{\vb{f}(x,y)}
	&=
	\pmqty{
		\lambda
		-
		\gammar
		\bar{y}^2
		&
		0
		\\
		&
		0
	}
\end{align}

\end{document} 
